% Nejprve uvedeme tridu dokumentu s volbami
\documentclass[slovak,bachelorpractice,dept460,male,csharp,cpdeclaration]{diploma}
% Dalsi doplnujici baliky maker
\usepackage[autostyle=true,czech=quotes]{csquotes} % korektni sazba uvozovek, podpora pro balik biblatex
\usepackage[backend=biber, style=iso-numeric, alldates=iso]{biblatex} % bibliografie
\usepackage{dcolumn} % sloupce tabulky s ciselnymi hodnotami
\usepackage{subfig} % makra pro "podobrazky" a "podtabulky"

% Zadame pozadovane vstupy pro generovani titulnich stran.
\ThesisAuthor{Miroslav Kačeriak}
\SubmissionDate{\today}

\Thanks{\textcolor{red}{Rád bych na tomto místě poděkoval všem, kteří mi s prací pomohli, pretože bez nich by tato práce nevznikla.}}


% Zadame cestu a jmeno souboru ci nekolika souboru s digitalizovanou podobou zadani prace.
% Pokud toto makro zapoznamkujeme sazi se stranka s upozornenim.
%\ThesisAssignmentImagePath{Figures/Assignment}

% Zadame soubor s digitalizovanou podobou prohlaseni autora zaverecne prace.
% Pokud toto makro zapoznamkujeme sazi se cisty text prohlaseni.
%\AuthorDeclarationImageFile{Figures/AuthorDeclaration.jpg}


% Zadame soubor s digitalizovanou podobou souhlasu spolupracujici prav. nebo fyz. osoby.
% Pokud toto makro zapoznamkujeme sazi se cisty text souhlasu.
%\CooperatingPersonsDeclarationImageFile{Figures/CoopPersonDeclaration.jpg}


\CzechAbstract{\textcolor{red}{Tohle je český abstrakt, zbytek odstavce je tvořen výplňovým textem. Naší si rozmachu potřebami s posílat v poskytnout ty má plot. Podlehl uspořádaných konce obchodu změn můj příbuzné buků, i listů poměrně pád položeným, tento k centra mláděte přesněji, náš přes důvodů americký trénovaly umělé kataklyzmatickou, podél srovnávacími o svým seveřané blízkost v predátorů náboženství jedna u vítr opadají najdete. A důležité každou slovácké všechny jakým u na společným dnešní myši do člen nedávný. Zjistí hází vymíráním výborná.}}

\CzechKeywords{typografie; \LaTeX; diplomová práce}

\EnglishAbstract{\textcolor{red}{This is English abstract. Lorem ipsum dolor sit amet, consectetuer adipiscing elit. Fusce tellus odio, dapibus id fermentum quis, suscipit id erat. Aenean placerat. Vivamus ac leo pretium faucibus. Duis risus. Fusce consectetuer risus a nunc. Duis ante orci, molestie vitae vehicula venenatis, tincidunt ac pede. Aliquam erat volutpat. Donec vitae arcu. Nullam lectus justo, vulputate eget mollis sed, tempor sed magna. Curabitur ligula sapien, pulvinar a vestibulum quis, facilisis vel sapien. Vestibulum fermentum tortor id mi. Etiam bibendum elit eget erat. Pellentesque pretium lectus id turpis. Nulla quis diam.}}

\EnglishKeywords{typography; \LaTeX; master thesis}

\AddAcronym{QA}{Quality Assurance}
\AddAcronym{PC}{Personal Computer}
\AddAcronym{OS}{Operating System}
\AddAcronym{RPG}{Role Playing Game}
\AddAcronym{EOS}{Epic Online Services}
\AddAcronym{CLR}{Common Language Runtime}
\AddAcronym{JIT}{Just-In-Time}
\AddAcronym{IL}{Intermediate Language}
\AddAcronym{XML}{eXtensible Markup Language}
\AddAcronym{WPF}{Windows Presentation Foundation}
\AddAcronym{LFS}{Large File Storage}
\AddAcronym{CI}{Continuous Integration}
\AddAcronym{CD}{Continuous Delivery}
\AddAcronym{CD}{Continuous Deployment}
\AddAcronym{AI}{Artificial intelligence}
\AddAcronym{REST}{Representational State Transfer}
\AddAcronym{API}{Application Programming Interface}
\AddAcronym{JSON}{JavaScript Object Notation}
\AddAcronym{TCP}{Transmission Control Protocol}
\AddAcronym{IP}{Internet Protocol}
\AddAcronym{DLL}{Dynamic Link Library}

\addbibresource{literature.bib}

% Novy druh tabulkoveho sloupce, ve kterem jsou cisla zarovnana podle desetinne carky
\newcolumntype{d}[1]{D{,}{,}{#1}}

\begin{document}
\MakeTitlePages

% A nasleduje text zaverecne prace.
\section{Úvod}
\label{sec:Introduction}
V rámci mojej odbornej praxe som dostal možnosť nahliadnuť za oponu herného vývoja v českom nezávislom štúdiu Perun Creative. Nakoľko sa o hry a herný priemysel dlhodobo zaujímam, táto firma a jej tvorba mi bola vopred známa. Aj po prednáške jej dvoch spoluzakladateľov a zároveň programátorov, ktorá sa uskutočnila v priestoroch Vysokej školy báňskej, som bol stále prekvapený vysokou technologickou úrovňou ich prvého projektu. Firmu som chcel kontaktovať so žiadosťou o prácu nezávisle od odbornej praxe, no keď som sa dozvedel, že už niekoho práve na odbornú prax hľadajú, rozhodol som sa to využiť. Pracovnému pohovoru predchádzalo zaslanie programátorského portfólia zloženého zo školských ale aj vlastných prác. Samotný pohovor potom prebiehal online s oboma programátormi Bc. Jánom Polachom a Bc. Jirkou Vašicou, ktorí sa neskôr stali aj mojimi kolegami.

Do firmy som nastúpil na konci životného cyklu projektu, takže som sa nemohol podieľať na vývoji základných herných mechaník. Naopak to znamenalo nutnosť dôkladne sa s celým projektom zoznámiť a pochopiť, ako jednotlivé veci fungujú. Počas praxe som sa podieľal na širokej škále väčších aj menších projektov z oblastí ako sú automatizácia a zefektívnenie vývojárskych či testerských postupov, nasadenie projektu, sieťová infraštruktúra a došlo aj na nejaké tie herné mechaniky. Pri riešení rôznych problémov mi okrem kolegov boli nápomocné aj rôzne teoretické znalosti nadobudnuté počas vysokoškolského štúdia.

% Section 2
\section{Popis firmy a pracovné zaradenie}
\label{sec:Firm and me}
% TODO: Overit info u Jirky
\subsection{Popis firmy a vyvíjaného produktu}
\label{sec:Firm}
Perun Creative s.r.o \cite{Perun} je české nezávislé herné štúdio, ktoré od roku 2015 vyvíja počítačovú hru Hobo: Tough Life \cite{Hobo}.

Hra samotná by sa dala charakterizovať ako RPG z mestského prostredia, kde sa hráč ocitne v role bezdomovca. Ústrednou hernou mechanikou je snaha prežiť v nehostinnom prostredí ulice. Okrem prežitia na hráča čaká aj pútavý príbeh a možnosť hrať kooperatívne až s troma ďalšími hráčmi. Hobo: Tough Life v súčasnosti vychádza na platformách Microsoft Windows a Linux. V budúcnosti sa počíta aj s vydaním na konzolách novej a starej generácie. 

Štúdio Perun Creative má aktuálne dve pobočky. Prvá sa nachádza v Ostrave a jej osadenstvo tvoria výhradne programátori. Pobočka v Olomouci naopak slúži pre menej technicky zameranú časť firmy a síce pre grafika, herného dizajnéra a komunitného manažéra. Štúdio tvorí menej ako desať vývojárov a teda sa svojou veľkosťou radí k menším. Využíva ale aj služby externých pracovníkov prípadne spoločností špecializovaných na testovanie, zvukovú stránku hry a v neposlednom rade aj na preklad textov do rôznych svetových jazykov.
\vspace{-20pt}
\begin{figure}[!htbp]
	\centering
	\includegraphics[width=0.5\textwidth]{Pictures/perunLogo.pdf}
	\vspace{-35pt}
	\caption{Logo spoločnosti Perun Creative}
	\label{pic:perunLogo}
\end{figure}
\vspace{-20pt}
\subsection{Pracovné zaradenie}
\label{sec:Me}
Môj prínos štúdiu Perun Creative spočíval hlavne v automatizácií a zefektívnení jednotlivých interných postupov resp. vývoji pomocných nástrojov pre QA oddelenie či ostatných programátorov. Do tejto kategórie by som zaradil projekty ako automatizované zostavenie hry na serveri, zefektívnenie spôsobu nahlasovania jednotlivých problémov prípadne prepracovanie importovania objektov zo starej verzie projektu do novej. K zefektívneniu práce rozhodne prispelo aj vytvorenie systému klávesových skratiek v hernom engine Unity (bližšie popísanom v sekcií \ref{sec:Unity}), ladenie zostavenej verzie hry zo vzdialeného PC prípadne program na vytvorenie tzv. \mbox{\enquote{Release notes}}. Ten bol primárne určený pre testerov, ale do budúcna sa plánuje jeho využitie aj pri informovaní hráčov o novinkách v rámci hry.

Okrem vyššie uvedených projektov som pracoval na novom systéme líhania postavy k spánku, optimalizácií a uvedeniu hry na OS Linux, prevode online časti hry z platformy Steam na platformu EOS a ďalších menších projektoch.
% End of section 2

% Section 3
\section{Použité technológie}
\label{sec:Tech}
V nasledujúcej kapitole by som rád v krátkosti zhrnul najdôležitejšie technológie, s ktorými som sa v rámci odbornej praxe stretol. Niektoré som aktívne využíval počas celého obdobia praxe a ich osvojovanie teda prebiehalo organicky. Iné boli špecifické pre konkrétny projekt a danú technológiu som si musel naštudovať počas realizácie projektu. V niektorých prípadoch bolo nutné sa zoznámiť s viacerými technológiami, aby som bol schopný posúdiť ich výhody a nevýhody pri nasadení na konkrétny projekt a vybrať tú správnu. Informácie som poväčšine čerpal z dokumentácií k daným technológiám aj keď nie vždy bola ich úroveň dostatočná.
\subsection{Unity Engine}
\label{sec:Unity}
Unity Engine \cite{Unity} je platforma pre tvorbu 2D a 3D interaktívneho obsahu renderovaného v reálnom čase. Najväčšie uplatnenie nachádza pri tvorbe malých až stredne veľkých hier, ale je čoraz častejšie využívaný aj v iných odvetviach ako napríklad automobilový či filmový priemysel. Je dostupný pre operačné systémy Windows, Linux a Mac OS. Distribuuje sa zdarma pre študentov alebo jednotlivcov do určitého finančného obratu a za ročný poplatok pre firmy, ktorého výška závisí od rôznych faktorov. Samotný engine je napísaný v jazyku C++ a umožňuje vývoj obsahu v jazyku C\# (bližšie popísanom v sekcií \ref{sec:CsharpDotNet}) pre širokú škálu platforiem, čo znázorňuje \mbox{obrázok \ref{pic:UnityPlatforms}}.

\begin{figure}[!htbp]
	\centering
	\includegraphics[width=1\textwidth]{Pictures/platforms.png}
	\caption{Platformy podporované Unity engine-om \cite{UnityMultiplatform}}
	\label{pic:UnityPlatforms}
\end{figure}

\subsection{Programovací jazyk C\# a architektúra .NET}
\label{sec:CsharpDotNet}
C\# \cite{CSharpLang} je objektovo orientovaný, typovo bezpečný programovací jazyk umožňujúci vytvárať aplikácie v .NET ekosystéme. Syntax jazyka C\# vychádza z programovacích jazykov C a C++. Narozdiel od nich ale ponúka vyššiu úroveň abstrakcie. Tá sa prejavuje napríklad na úrovni správy pamäte, čo z jazyka C\# robí voľbu číslo jedna pre začínajúcich programátorov, ktorí by si chceli skúsiť herný vývoj na vlastnej koži.

Architektúra .NET \cite{CSharpLang} potom ponúka okrem virtuálneho stroja CLR, ktorý vykonáva JIT kompiláciu IL kódu do strojových inštrukcií, aj sadu veľmi užitočných knižníc. Tieto knižnice sú organizované do menných priestorov a poskytujú širokú škálu metód vhodných napríklad na prácu so súbormi či sieťovou infraštruktúrou. Veľmi užitočné sú aj nástroje na syntaktickú analýzu XML prípadne platforma Windows Forms, ktorá sa v rámci mojej odbornej praxe vo firme Perun Creative ukázala byť vhodná hlavne na rýchlu tvorbu vývojárskych nástrojov. Kvôli limitáciám tejto technológie sa ale v budúcnosti plánuje prechod na systém WPF.
%pridaj aj winforms
\subsection{Verzovací systém Git}
\label{sec:Git}
Git \cite{ProGit} je open source distribuovaný systém na správu verzií vyvinutý Linusom Thorvaldsom a komunitou okolo OS Linux v roku 2005. Umožňuje vrátiť celý projekt alebo vybrané časti do predchádzajúceho stavu, porovnávať zmeny v súboroch, prípadne efektívnu kolaboráciu viacerých vývojárov na spoločnom projekte. Každý člen tímu má k dispozícií kompletný repozitár vrátane histórie jednotlivých súborov, čo pridáva ďalšiu vrstvu ochrany proti výpadkom a poruchám hardvéru. Po každom spustení príkazu \textit{commit} sa uloží aktuálny stav projektu, ktorý je možné spätne dohľadať v prípade nejakého problému.

S gitom je možné pracovať priamo z príkazového riadku ale pre väčšie projekty ako je aj hra Hobo: Tough Life je takýto prístup značne nepraktický a neefektívny. Existuje ale množstvo programov tretích strán, ktoré umožňujú pracovať s gitom z užívateľského rozhrania. Jedným takým je aj nástroj Sourcetree, s ktorým som v rámci mojej odbornej praxe pracoval.

Nakoľko git samotný nie je dobre prispôsobený na správu verzií veľkých súborov, bolo nutné využívať aj jeho rozšírenie Git LFS. 
\subsection{GitLab CI/CD}
\label{sec:GitLab}
Gitlab CI/CD \cite{Cicd} je súčasť nástroja Gitlab a slúži na vývoj softvéru prostredníctvom kontinuálnych metód ako:
\begin{itemize}
  \item \textbf{Continuous Integration (CI)} - pre každé pridanie zmien do repozitára je možné automatizovane spustiť sadu skriptov, ktorých účelom je projekt zostaviť či otestovať.
  \item \textbf{Continuous Delivery (CD)} - pridáva ďalšiu vrstvu nad rámec CI, umožňuje zostavený a otestovaný kód nasadiť, vyžaduje to ale manuálnu akciu.
  \item \textbf{Continuous Deployment (CD)} - funguje podobne ako Continuous Delivery ale nasadenie prebieha automaticky.
\end{itemize}
Spoločnou filozofiou týchto metód je teda po každej iteratívnej zmene v projekte kód zostaviť, otestovať a prípadne aj nasadiť. Hlavnou výhodou takého prístupu je redukovanie množstva chýb, ktoré by sa inak dostávali do ďalších iterácií a mohli by spôsobiť problémy, keby neboli odchytené v zárodku. Zároveň si kladú za cieľ znížiť potrebu manuálneho zásahu do jednotlivých automatizovaných procesov na minimum.

V rámci mojej odbornej praxe som sa stretol hlavne s metódou CI. Možnosti automatizovaného testovania hry touto metódou sú veľmi obmedzené, o to viac je ale užitočné automatizované zostavenie na serveri. Zostavenie tohto typu projektu na pracovnej stanici je typicky náročne ako časovo tak aj na výpočtový výkon a do značnej miery teda spomaľuje vývoj ako taký. Bolo teda logickým krokom zamerať sa v rámci optimalizácie interných procesov aj na tento aspekt.

\begin{figure}[!htbp]
	\centering
	\includegraphics[width=1\textwidth]{Pictures/gitlab.png}
	\caption{Princíp Gitab CI/CD \cite{Cicd}}
	\label{pic:Gitlab}
\end{figure}

\subsection{Epic Online Services}
\label{sec:Eos}
Epic Online Services alebo EOS \cite{Epic} je skupina nástrojov od spoločnosti Epic Games, ktoré sa snažia zjednodušiť a zjednotiť niektoré aspekty herného vývoja naprieč platformami. Typicky sa zameriavajú na kooperatívnu či kompetetívnu zložku hry pre viacerých hráčov, zahŕňajú však aj služby ako nákup herných predmetov za reálne peniaze, zber hráčskych štatistik, prípadne získavanie herných úspechov. Za normálnych okolností je nutné tieto služby implementovať zvlášť takmer pre každú platformu, na ktorú bude hra vychádzať. EOS sú zdarma dostupné pre vývojárov a podporujú širokú škálu platforiem, čo znamená väčšiu slobodu z pohľadu hráča ale aj z pohľadu vývojára. Hráč môže ďalej ťažiť aj z online synchronizácie postupu hrou naprieč viacerými platformami, prípadne z prístupu k úspechom či priateľom z jedného centrálneho miesta. 

Nakoľko sa do budúcna počíta s vydaním hry Hobo: Tough Life na rôzne platformy, ukázalo sa použitie EOS ako veľmi výhodné na náhradu doterajšieho systému pre hru viacerých hráčov. Hlavnou výhodou tohto riešenia by bola vyššie spomínaná multiplatformovosť. V dobe písania tohto textu sa však počíta s nasadením EOS spolu s niektorými službami dostupnými len na konkrétnej platforme.
% End of section 3

% Section 4
\section{Zadané projekty a ich riešenia}
\label{sec:Projects}
V tejto kapitole by som rád podrobne prebral konkrétne projekty na ktorých som v rámci mojej odbornej praxe pracoval. Poradie v akom budú prezentované nutne neodzrkadľuje poradie ich vypracovania, ale budú usporiadané do logických celkov podľa typu projektu. 

Čo sa časovej náročnosti týka, nie vždy je možné ju určiť presne nakoľko niektoré z projektov boli vyvíjané inkrementálne podľa spätnej väzby QA oddelenia alebo nových požiadaviek od ostatných programátorov.
%U niektorych prebiehala tvorba inkrementalne a bola zavisla na spatnej vazbe bla bla.

\subsection{Vývoj pomocných nástrojov pre QA oddelenie a programátorov}
\label{sec:QACode}
\subsubsection{Evidovanie chybových reportov (Mantis Report)}
\label{sec:Report}
\textbf{Časová náročnosť:} \\ Nasadenie: 8 dní, pridanie ďalšej funkcionality podľa požidaviek: spolu 7 dní.\\
\textbf{Úvod do problému:} \\ Ako som spomenul v sekcií \ref{sec:GitLab} možnosti automatizovaného testovania hry tohto rozsahu sú na rozdiel od bežného softvéru značne obmedzené. Veľké štúdiá si vytvárajú vlastné nástroje založené na AI, nikdy sa to však úplne neobíde bez tvrdej práce ľudí z QA oddelenia ľudovo nazývaných aj ako testeri. Úlohou testera je teda hrať celú hru alebo jej určené časti, nájsť a následne ohlásiť nájdené chyby. 

Takéto ohlásenie chyby je značne neefektívny proces nakoľko na pozadí zahŕňa hneď niekoľko ďalších krokov ako minimalizácia hry, otvorenie stránky so službou Mantis Bug Tracker, vloženie problému a nakoniec aj samotné vypĺňanie detailov reportu, ktoré znázorňuje obrázok \ref{pic:Mantis}.
Táto neefektívnosť samozrejme priamo úmerne rastie s počtom nájdených chýb, čo môžu byť aj vyššie jednotky denne. Mojou úlohou bolo teda optimalizovať tento postup. \\
\textbf{Navrhované riešenia:} \\ Nakoľko bola toto moja prvá úloha v rámci odbornej praxe, mal som len minimálne skúsenosti s webovými technológiami ako je REST API a celkovo klient-server architektúrou, navrhol som riešenie postavené na nejakom nástroji na automatizovanie webového prehliadača. Takýmto nástrojom je napríklad Selenium WebDriver, s ktorým som sa stretol pri práci na vlastných projektoch. Ten je veľmi obľúbený napríklad pri testovaní webových aplikácií. Dokázal by zapnúť webový prehliadač, či už s užívateľským rozhraním alebo bez, otvoriť požadovanú stránku a vyplniť detaily reportu za užívateľa. V spojení s nejakým vlastným nástrojom na ukladanie snímky obrazovky, prípadne nástrojom, ktorý by testerovi umožnil vypísať zhrnutie či popis reportu priamo v hre by sa naozaj jednalo o relatívne dobré riešenie. Po užívateľovi by to ale vyžadovalo nutnosť inštalácie nejakého konkrétneho prehliadača v požadovanej verzií, aby bola zaistená správna kompatibilita a veľmi pravdepodobne by sa v budúcnosti objavili aj ďalšie problémy s nasadením či používaním. 

Po preskúmaní ďalších možností a následnej porade s kolegami som sa teda rozhodol dať prednosť riešeniu postavenému na už spomínanom REST API a ak by sa to ukázalo ako nerealizovateľné spätne sa vrátiť k môjmu prvému nápadu.
\begin{figure}[!htbp]
	\centering
	\includegraphics[width=0.7\textwidth]{Pictures/mantis.png}
	\caption{Ručné vloženie reportu do služby Mantis Bug Tracker}
	\label{pic:Mantis}
\end{figure} \\
\textbf{Realizácia:} \\ Pri preskúmavaní možností založených na REST API som narazil na open source projekt \mbox{MantisSharp \cite{MantisSharp}.} Po zoznámení sa s týmto projektom a niekoľkými pokusmi ho \enquote{ohnúť} pre účely môjho projektu som sa rozhodol, že bude jednoduchšie napísať si aplikáciu sám. Využil som k tomu dve triedy z projektu MantisSharp, a síce RestClient a MantisClient, ktoré som následne ešte ďalej modifikoval. Trieda RestClient vykonáva najnižšiu úroveň komunikácie so serverom a síce odosiela GET a POST žiadosti. Trieda MantisClient má potom dve úlohy. Na jednej strane prijíma dáta z mojej aplikácie, zabalí ich do formy vhodnej pre transport a následne ich predá triede \mbox{RestClient} na odoslanie metódou POST. Na strane druhej transformuje dáta získané zo servera pomocou metódy GET do C\# tried vhodných na následné použitie. Tento postup pridáva určitý level abstrakcie do celej aplikácie.

Problematiku odosielania dát na server som sa rozhodol demonštrovať na pridaní nového reportu vo výpise \ref{src:SendIssue}. Aplikácia vytvorí na základe vstupných dát inštanciu triedy Issue, ktorú predá metóde SendIssue. Tá ju následne zabalí ako JSON objekt a ďalej predá v metóde ExecutePost na finálne odoslanie. Získavanie dát funguje obdobne len opačným smerom. 

Služba Mantis Bug Tracker po úspešnom pridaní nového reportu tento report vráti v odpovedi, čo som ďalej využil na informovanie užívateľa o úspešnom zaevidovaní, ktoré som doplnil o serverom pridelený identifikátor.
\vspace{10pt}
\begin{lstlisting}[label=src:SendIssue,caption={Metóda SendIssue triedy MantisClient}]
public int SendIssue(Issue issue)
{
    string uri = this.GetUri(issuesUri);
    int id = -1;

    restClient.ExecutePost(uri, () => JsonConvert.SerializeObject(issue),
        reader =>
        {
            string answerFromServer = reader.ReadToEnd();
            JObject obj = JObject.Parse(answerFromServer);
            id = (int)obj["issue"]["id"];
        });

    return id;
}
\end{lstlisting}
Server spočiatku na všetky žiadosti reagoval chybou \enquote{401 Unauthorized}. Po dôkladnom preskúmaní problému sa ukázalo, že chyba je na strane serveru a bolo nutné zasiahnuť do jeho kódu. Nakoľko som k tomuto kódu nemal prístup, požiadal som kolegu, či by mi s tým mohol pomôcť. Ukázalo sa, že server filtroval všetky požiadavky, ktoré sa pokúšali o autorizáciu pomocou tzv. \enquote{Bearer tokenu}. Spoločne sa nám tento problém však podarilo vyriešiť. 

Systém nahlasovania reportov mal byť pôvodne implementovaný ako súčasť hry. To sa ale ukázalo ako problematické z hľadiska zachytávania užívateľského vstupu v Unity engine. V praxi by to znamenalo, že ak by užívateľ popisoval chybu napríklad do nejakého textového poľa, hra samotná by naďalej vykonávala akcie na základe stlačených kláves. Rozhodli sme sa teda začleniť tento systém do už existujúceho vývojárskeho nástroja s názvom HoboThor. Tento nástroj je veľmi komplexný a jeho popis by bol nad rámec tejto kapitoly.

Logika aplikácie bola teda rozdelená do dvoch častí. Časť, s ktorou interaguje užívateľ bola realizovaná ako Windows Forms aplikácia začlenená do nástroja HoboThor a časť, ktorá túto aplikáciu spustí bola implementovaná priamo do hry a vyvolá stlačením klávesovej skratky Shift + F11, ak je hra spustená s podporou vývojárskych nástrojov. 

Kontrola, či boli stlačené konkrétne klávesy musí prebiehať každú snímku, aby sa zabezpečilo, že odchytenie prebehne úspešne. Kód zabezpečujúci túto funkcionalitu bol teda vložený do metódy OnUpdate triedy ReportingManager, čo znázorňuje výpis \ref{src:OnUpdate}. Projekt Hobo: Tough Life do veľkej miery stojí na hierarchickej štruktúre tried, tzv. \enquote{Manageroch}, ktorí, ako názov napovedá obstarávajú určité časti aplikácie. Metóda OnUpdate triedy ReportingManager je teda každú snímku volaná inou triedou, ktorá je vyššie v hierarchickej štruktúre projektu. Na najvyššom stupni hierarchie potom stojí trieda MainManager. Tá obsahuje metódu Update, ktorá je priamo volaná natívnym C++ kódom enginu Unity a táto metóda potom obsahuje volania metód OnUpdate jednotlivých \enquote{Managerov}, ktorí jej náležia.
\vspace{10pt}
\begin{lstlisting}[label=src:OnUpdate,caption={Metóda OnUpdate triedy ReportingManager}]
public void OnUpdate()
{
    if (Keyboard.current.f11Key.wasPressedThisFrame)
    {
        if (Keyboard.current.leftShiftKey.isPressed)
        {
            screenSaved = false;
            StartCoroutine(StartCreateReport());
            return;
        }
    }

    if (screenSaved)
    {
        ZipAndSendToLinkManager();
        screenSaved = false;
    }
}
\end{lstlisting}

Kód vo výpise \ref{src:OnUpdate} teda každú snímku testuje, či bola stlačená klávesa F11 a to práve v tom danom snímku. Toto zabezpečí, že sa blok kódu tejto podmienky vykoná len raz, bez ohľadu na to, ako dlho užívateľ danú klávesu držal stlačenú. Ak je popri tom stlačená aj klávesa Shift, spustí sa tzv. \enquote{Coroutine}. V tomto prípade je ňou mnou definovaná metóda StartCreateReport. Výhoda korutiny spočíva v možnosti pozastaviť vykonávanie svojho kódu. Pozastavenie môže byť na programátorom definovanú dobu alebo do doby než nastane určitá udalosť. V tomto prípade bolo tou udalosťou kompletné vykreslenie aktuálneho snímku a to z dôvodu zachytenia tohto snímku ako obrázku.

Metóda StartCreateReport má za úlohu zozbierať rôzne dáta o aplikácií alebo systéme, na ktorom je spustená. Ide o dáta ako verzia aplikácie, pozícia a rotácia kamery v momente vyvolania akcie či posledný záznam v logovacom systéme. Tieto dáta sú následne spolu so spomínanou snímkou obrazovky a naposledy uloženým postupom hrou uložené na disk. Zložka, do ktorej sú súbory uložené závisí od toho, či je hra spustená z editoru Unity alebo samostatne. Takéto vetvenie kódu sa realizuje pomocou direktívy UNITY\_EDITOR, ktorá je spolu s ďalšími platformovo závislými direktívami definovaná samotným editorom. Jednoduchý príklad použitia takýchto direktív demonštruje výpis \ref{src:Pragma}. Nakoľko táto zložka obsahuje celú históriu reportov, jednotlivé reporty majú poradové čísla a každý ďalší dostane pridelené poradové číslo o jedna väčšie ako ten predchádzajúci.
\vspace{10pt}
\begin{lstlisting}[label=src:Pragma,caption={Ukážka použitia direktívy UNITY\_EDITOR}]
#if UNITY_EDITOR
	Debug.Log("Editor");                
#else
    Debug.Log("Standalone");             
#endif
\end{lstlisting}

Následne sa logika aplikácie delí na dve vetvy. Prvá obstaráva vytvorenie lokálneho reportu, druhá vzdialeného. Možnosť vzdialeného nahlasovania chýb, teda odoslanie reportu z PC na ktorom hra nie je spustená bola pridaná až neskôr v súvislosti s prevodom hry na iné platformy. Oba spôsoby odoslania reportu potom demonštruje výpis \ref{src:Report}.
\vspace{10pt}
\begin{lstlisting}[label=src:Report,caption={Vytváranie lokálneho a vzdialeného reportu}]
// Local report
if (HBTLink_Manager.sender_ServerIP == HBTLink_Manager.Instance.LocalIPAddress().ToString())
{
    string exeFile = GameConfiguration.pathConfig.exeFile;
    if (File.Exists(exeFile))
    {
        var argFile = GameConfiguration.pathConfig.mantisArguments;

        if (File.Exists(argFile))
            File.Delete(argFile);

        File.WriteAllText(argFile, directoryReport.FullName);
        Application.OpenURL(exeFile);
    }
}
// Remote report
else
{
    float time = 0;
    while (time <= maxWaitTime)
    {
        yield return new WaitForSeconds(.1f);
        time++;
        if (File.Exists(screenFileName))
        {
            screenSaved = true;
            yield break;
        }
    }
}
\end{lstlisting}

Odoslanie reportu na server z lokálneho PC pokračuje vytvorením tzv. \enquote{Mantis argumentu}. Ide o textový súbor, ktorého obsah je cesta k naposledy vytvorenému reportu. Ten je uložený do zložky, kde sa nachádza spúšťací súbor nástroja HoboThor. Následne je tento nástroj spustený a pokiaľ je pri tomto procese  prítomný už spomínaný súbor, nespustí sa hlavné okno aplikácie ale len okno určené na odoslanie reportu. Po prečítaní obsahu je súbor samozrejme zmazaný, aby ďalšie spustenie nástroja HoboThor bolo plnohodnotné.

Pokiaľ ide o odoslanie reportu zo vzdialeného PC ukázala sa ako problematická doba ukladania snímky obrazovky na disk. Kvôli tomuto problému musí aplikácia najskôr počkať na uloženie súboru a až následne vykonávať ďalšie inštrukcie. Toho som docielil jednoduchou slučkou, ktorá sa sama ukončí v prípade nájdenia požadovaného súboru alebo po pretečení určitého času. To slúži ako ochrana pred zacyklením. Na premennú \textit{screenSaved} zareaguje metóda OnUpdate z výpisu \ref{src:OnUpdate} a spustí metódu ZipAndSendToLinkManager. Tá pomocou knižnice DotNetZip \cite{DotNetZip} prevedie celú zložku s reportom na zip súbor so zachovaním hierarchickej štruktúry podzložiek a súborov. Následne ho prevedie na reťazec s kódovaním base64 a predá triede HBTLink\_Manager na odoslanie do vzdialeného PC. Táto metóda je znázornená vo výpise \ref{src:Zip}. Rreťazec je následne pomocou TCP protokolu odoslaný na IP adresu určenú v konfiguračnom súbore alebo vybratú z prednastavených možností vo vývojárskej konzole počas hrania. Na vzdialenom PC musí byť spustený nástroj HoboThor a \enquote{počúvať} na určenom porte. Dáta sú po prijatí uložené do dočasnej zložky poskytovanej operačným systémom a následne rozbalené. Cesta k tejto zložke je opäť zapísaná do \enquote{Mantis argumentu} a HoboThor je následne automaticky spustený znova. To vyústi k otvoreniu okna určeného na nahlasovanie chýb a načítaniu dát zo zložky s reportom.
\vspace{20pt}
\begin{lstlisting}[label=src:Zip,caption={Metóda ZipAndSendToLinkManager triedy ReportingManager}]
private void ZipAndSendToLinkManager()
{
    string pathToZip = directoryReport.FullName + @"\report.zip";

    using (ZipFile zip = new ZipFile())
    {
        var files = Directory.EnumerateFiles(directoryReport.FullName, "*.*", SearchOption.AllDirectories);
        foreach (string file in files)
        {
            if (file.Contains("Characters"))
                zip.AddFile(file, @"\Saves\Characters");
            else if (file.Contains("Worlds"))
                zip.AddFile(file, @"\Saves\Worlds");
            else
                zip.AddFile(file, "");
        }

        zip.Save(pathToZip);
        Debug.Log("Zipped to " + pathToZip);
    }

    if (File.Exists(pathToZip))
    {
        string base64zip = System.Convert.ToBase64String(File.ReadAllBytes(pathToZip));
        HBTLink_Manager.Send_MantisReport(base64zip);
        File.Delete(pathToZip);
    }
}
\end{lstlisting}

Mojou poslednou úlohou v rámci tejto časti reportovacieho systému, ktorá prišla ako požiadavka od programátorov bolo pridať možnosť odosielať reporty priamo z editoru bez nutnosti mať spustenú hru. Typický use case bolo nahlasovanie chýb nájdených popri práci s náhľadom scény. Väčšina  kódu bola prebratá z vyššie popísanej časti, problém ale nastal so snímkou obrazovky. Pôvodne použitá metóda CaptureScreenshotAsTexture funguje iba v rámci spustenej hry, nie v editore. Tú som teda nahradil metódou ReadScreenPixel, ktorá ale potrebuje vedieť presnú lokáciu a rozmery okna. Získať tieto údaje sa mi nakoniec podarilo až volaním C++ metód z user32.dll. Následným testovaním na 4K monitore sa ukázalo, že tento postup nevhodne vyhodnocuje veľkosť okna s nastaveným škálovaním v OS Windows. Toto bolo nutné upraviť ručne, čo ukazuje výpis \ref{src:DLL}.
\vspace{10pt}
\begin{lstlisting}[label=src:DLL,caption={Získanie veľkosti okna Unity editoru}]
[DllImport("user32.dll")]
private static extern bool GetWindowRect(IntPtr hwnd, ref Rect rectangle);

[DllImport("user32.dll")]
private static extern IntPtr GetActiveWindow();

private struct Rect
{
    public int Left { get; set; }
    public int Top { get; set; }
    public int Right { get; set; }
    public int Bottom { get; set; }
}

private static Rect GetUnityBounds()
{
    Rect unityWindowBounds = new Rect();
    float scale = Screen.dpi / 96;

    GetWindowRect(GetActiveWindow(), ref unityWindowBounds);

    unityWindowBounds.Right = (int)(unityWindowBounds.Right / scale);
    unityWindowBounds.Bottom = (int)(unityWindowBounds.Bottom / scale);

    return unityWindowBounds;
}
\end{lstlisting}

Na strane Windows Forms aplikácie implementovanej do nástroja HoboThor bolo nutné najskôr vytvoriť sadu tried a enumov, ktoré by po serializácií/deserializácií presne kopírovali štruktúru projektu či reportu s ktorou pracuje služba Mantis Bug Tracker. Následne bolo nutné vytvoriť užívateľské rozhranie, ktoré ale postupne prechádzalo iteratívnymi zmenami až do svojej finálnej podoby znázornenej na obrázku \ref{pic:Report}.
\begin{figure}[!htbp]
	\centering
	\setlength{\fboxsep}{0pt}
	\setlength{\fboxrule}{1pt}
	\fbox {
		\includegraphics[width=1\textwidth]{Pictures/report.png}
	}
	\caption{Užívateľské rozhranie aplikácie Mantis Report}
	\label{pic:Report}
\end{figure}
%Mantis form a picture edit TODO
\subsubsection{Vytváranie \enquote{Release notes} pre QA oddelenie (GitLogger)}
\label{sec:GitLogger}
\subsubsection{Automatizované zostavenie hry na serveri (Build Server)}
\label{sec:BuildServer}
\subsection{Optimalizácia a prevod hry na ďalšie platformy}
\label{sec:Port}
\subsection{Ostatné projekty}
\label{sec:Others}
asd
% End of section 4

% Section 5
%\section{Postup riešenia zadaných projektov}
%\label{sec:Processes}
%adasd
% End of section 5

%d) Teoretické a praktické znalosti a dovednosti získané v průběhu studia uplatněné studentem v průběhu odborné praxe.
%e) Znalosti či dovednosti scházející studentovi v průběhu odborné praxe.
%f) Dosažené výsledky v průběhu odborné praxe a její celkové zhodnocení.

\printbibliography[title={Literatúra}, heading=bibintoc]
\end{document}
