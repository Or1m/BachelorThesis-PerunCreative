% Nejprve uvedeme tridu dokumentu s volbami
\documentclass[slovak,bachelorpractice,dept460,male,csharp,cpdeclaration]{diploma}
% Dalsi doplnujici baliky maker
\usepackage[autostyle=true,czech=quotes]{csquotes} % korektni sazba uvozovek, podpora pro balik biblatex
\usepackage[backend=biber, style=iso-numeric, alldates=iso]{biblatex} % bibliografie
\usepackage{dcolumn} % sloupce tabulky s ciselnymi hodnotami
\usepackage{subfig} % makra pro "podobrazky" a "podtabulky"

% Zadame pozadovane vstupy pro generovani titulnich stran.
\ThesisAuthor{Miroslav Kačeriak}
\SubmissionDate{\today}

\Thanks{\textcolor{red}{Rád bych na tomto místě poděkoval všem, kteří mi s prací pomohli, pretože bez nich by tato práce nevznikla.}}


% Zadame cestu a jmeno souboru ci nekolika souboru s digitalizovanou podobou zadani prace.
% Pokud toto makro zapoznamkujeme sazi se stranka s upozornenim.
%\ThesisAssignmentImagePath{Figures/Assignment}

% Zadame soubor s digitalizovanou podobou prohlaseni autora zaverecne prace.
% Pokud toto makro zapoznamkujeme sazi se cisty text prohlaseni.
%\AuthorDeclarationImageFile{Figures/AuthorDeclaration.jpg}


% Zadame soubor s digitalizovanou podobou souhlasu spolupracujici prav. nebo fyz. osoby.
% Pokud toto makro zapoznamkujeme sazi se cisty text souhlasu.
%\CooperatingPersonsDeclarationImageFile{Figures/CoopPersonDeclaration.jpg}


\CzechAbstract{\textcolor{red}{Tohle je český abstrakt, zbytek odstavce je tvořen výplňovým textem. Naší si rozmachu potřebami s posílat v poskytnout ty má plot. Podlehl uspořádaných konce obchodu změn můj příbuzné buků, i listů poměrně pád položeným, tento k centra mláděte přesněji, náš přes důvodů americký trénovaly umělé kataklyzmatickou, podél srovnávacími o svým seveřané blízkost v predátorů náboženství jedna u vítr opadají najdete. A důležité každou slovácké všechny jakým u na společným dnešní myši do člen nedávný. Zjistí hází vymíráním výborná.}}

\CzechKeywords{typografie; \LaTeX; diplomová práce}

\EnglishAbstract{\textcolor{red}{This is English abstract. Lorem ipsum dolor sit amet, consectetuer adipiscing elit. Fusce tellus odio, dapibus id fermentum quis, suscipit id erat. Aenean placerat. Vivamus ac leo pretium faucibus. Duis risus. Fusce consectetuer risus a nunc. Duis ante orci, molestie vitae vehicula venenatis, tincidunt ac pede. Aliquam erat volutpat. Donec vitae arcu. Nullam lectus justo, vulputate eget mollis sed, tempor sed magna. Curabitur ligula sapien, pulvinar a vestibulum quis, facilisis vel sapien. Vestibulum fermentum tortor id mi. Etiam bibendum elit eget erat. Pellentesque pretium lectus id turpis. Nulla quis diam.}}

\EnglishKeywords{typography; \LaTeX; master thesis}

\AddAcronym{DVD}{Digital Versatile Disc}
\AddAcronym{TNT}{Trinitrotoluen}
\AddAcronym{UML}{Unified Modeling Language}
\AddAcronym{HTML}{Hyper Text Markup Language}
\AddAcronym{TUG}{\TeX{} Users Group}

% Cesty k suborom s literarnymi zdrojmi, mozno nebude treba 
%\addbibresource{biblatex-examples.bib}
%\addbibresource{coffee.bib}

% Novy druh tabulkoveho sloupce, ve kterem jsou cisla zarovnana podle desetinne carky
\newcolumntype{d}[1]{D{,}{,}{#1}}

\begin{document}
\MakeTitlePages

% A nasleduje text zaverecne prace.
\section{Úvod}
\label{sec:Introduction}
V rámci mojej odbornej praxe som dostal možnosť nahliadnuť za oponu herného vývoja v českom nezávislom štúdiu Perun Creative. Nakoľko sa o hry a herný priemysel dlhodobo zaujímam, táto firma a jej tvorba mi bola vopred známa. Aj po prednáške jej dvoch spoluzakladateľov a zároveň programátorov, ktorá sa uskutočnila v priestoroch Vysokej Školy Báňskej, som bol stále prekvapený vysokou technologickou úrovňou ich prvého projektu. Firmu som chcel kontaktovať so žiadosťou o prácu nezávisle od odbornej praxe, no keď som sa dozvedel, že už niekoho práve na odbornú prax hľadajú, rozhodol som sa to využiť. Pracovnému pohovoru predchádzalo zaslanie programátorského portfólia zloženého zo školských ale aj vlastných prác. Samotný pohovor potom prebiehal online s oboma programátormi Bc. Jánom Polachom a Bc. Jirkou Vašicou, ktorí sa neskôr stali aj mojimi kolegami.

Do firmy som nastúpil na konci životného cyklu projektu, takže som sa nemohol podieľať na vývoji základných herných mechaník. Naopak to znamenalo nutnosť dôkladne sa s celým projektom zoznámiť a pochopiť, ako jednotlivé veci fungujú. Počas praxe som sa podieľal na širokej škále väčších aj menších projektov z oblastí ako sú automatizácia a zefektívnenie vývojárskych či testerských postupov, nasadenie projektu, sieťová infraštruktúra a došlo aj na nejaké tie herné mechaniky. Pri riešení rôznych problémov mi okrem kolegov boli nápomocné aj rôzne teoretické znalosti nadobudnuté počas vysokoškolského štúdia.

% Section 2
\section{Popis firmy a pracovné zaradenie}
\label{sec:Firm and me}
% TODO: Overit info u Jirky
\subsection{Popis firmy}
\label{sec:Firm}
Perun Creative s.r.o je české nezávislé herné štúdio, ktoré od roku 2015 vyvíja hru Hobo: Tough Life. Hra samotná by sa dala charakterizovať ako RPG z mestského prostredia.%ktoré má v súčasnosti 2 pobočky. Prvá sa nachádza v Ostrave
\subsection{Pracovné zaradenie}
\label{sec:Me}
% End of section 2
\end{document}
